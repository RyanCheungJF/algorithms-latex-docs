\documentclass[12pt,letterpaper]{article}
\usepackage{fullpage}
\usepackage[top=2cm, bottom=4.5cm, left=2.5cm, right=2.5cm]{geometry}
\usepackage{amsmath,amsthm,amsfonts,amssymb,amscd}
\usepackage{lastpage}
\usepackage{enumerate}
\usepackage{fancyhdr}
\usepackage{mathrsfs}
\usepackage{xcolor}
\usepackage{graphicx}
\usepackage{listings}
\usepackage{hyperref}
\usepackage{amsmath}
\usepackage{mathtools}
\usepackage{tikz}
\usepackage{array}
\usetikzlibrary{matrix}

\hypersetup{%
  colorlinks=true,
  linkcolor=blue,
  linkbordercolor={0 0 1}
}
 
\renewcommand\lstlistingname{Section}
\renewcommand\lstlistlistingname{Algorithms}
\def\lstlistingautorefname{Alg.}

\lstdefinestyle{Python}{
    language        = Python,
    frame           = lines, 
    basicstyle      = \footnotesize,
    keywordstyle    = \color{blue},
    stringstyle     = \color{green},
    commentstyle    = \color{red}\ttfamily
}

\setlength{\parindent}{0.0in}
\setlength{\parskip}{0.05in}

% Edit these as appropriate
\newcommand\course{CSE 3500}
\newcommand\hwnumber{2}                  % <-- homework number
\newcommand\NetIDa{rjf23002}           % <-- NetID of person #1
\newcommand\NetIDb{}           % <-- NetID of person #2 (Comment this line out for problem sets)

\pagestyle{fancyplain}
\headheight 35pt
\lhead{\NetIDa}
\lhead{\NetIDa\\\NetIDb}                 % <-- Comment this line out for problem sets (make sure you are person #1)
\chead{\textbf{\Large Homework \hwnumber}}
\rhead{\course \\ \today}
\lfoot{}
\cfoot{}
\rfoot{\small\thepage}
\headsep 1.5em

\begin{document}

\section*{Problem 0}

\begin{enumerate}
    \item
    Leaves at level k: $2^k$ \\
    Work done at top: O(n) \\
    Work done at level k: $2^k$ * O[(n/$2^k$)] = O(n)\\
    Work done across all levels: $log_2 n$ * O(n) \\
    Conclusion: Even work \\
    Bound: O(nlogn)
    \item
    Leaves at level k: $2^k$ \\
    Work done at top: O(1) \\
    Work done at level k: $2^k$ * O(1) = $O(2^k)$ \\
    Work done across all levels: $\sum^{log_2 n}_{i=1} O(2^k)$ = 2(n - 1) \\
    Conclusion: Bottom Heavy \\
    Bound: O(n)
     \item
    Leaves at level k: $7^k$ \\
    Work done at top: $O(n^3)$ \\
    Work done at level k: $7^k$ * O[$(n/{2^k})^3$] = $(7/8)^k$ * O($n^3$) $<$ Top level work done \\
    Conclusion: Top Heavy \\
    Bound: O($n^3$)
    \item
    Leaves at level k: $7^k$ \\
    Work done at top: $O(n^2)$ \\
    Work done at level k: $7^k$ * O[$(n/{2^k})^2$] = $(7/4)^k$ * O($n^2$) \\
    Work done across all levels: $\sum^{log_2n}_{i=1} (7/4)^k * O(n^2) = 7^{log_2 n}$ without constants \\
    Conclusion: Bottom Heavy \\
    Bound: O($n^{log_2 7}$)
    \item
    Leaves at level k: $4^k$ \\
    Work done at top: $O(n^{2.5})$ \\
    Work done at level k: $4^k$ * O[$(n/{2^k})^{5/2}$] = $2^{0.5k} * O(n^{2.5}) <$ Top level work done \\
    Conclusion: Top Heavy \\
    Bound: O($n^{2.5}$)
    \item
    Leaves at level k: $4^k$ \\
    Work done at top: $O(nlog_2 n)$ \\
    Work done at level k: $4^k$ * O[$(n/2^k)log(n/2^k)$] = O($2^knlogn$) without constants \\
    Work done across all levels: $\sum^{log_2n}_{i=1} 2^k * O(nlogn) = O(nlogn) * n $ without constants \\
    Conclusion: Bottom Heavy \\
    Bound: O($n^2logn$)
\end{enumerate}

\newpage

\section*{Problem 1}

We can solve this problem with divide and conquer by
splitting up a valid board into smaller multiple valid boards.
Given an arbitrary value k, we have a board of width and height $2^k$.
We notice that it is actually made up of 4 instances of a board $2^{k - 1}$. 
This splits up into 4 similar subproblems. 
Let us take a single subproblem (assume the subproblem we are looking into contains the removed cell) and continue breaking it down. 
We can continuously repeat this until we get to a base case where k = 1. \\

At k = 1, we have a 2x2 grid.
With a 2x2 grid, we can remove the top left and we are left with 3 remaining cells. 
These cells can be nicely filled with a single L.
Futhermore, we can also observe the other trivial base case of k = 0.
At k = 0, we only have a 1x1 grid.
Since the top left is removed, there is nothing to fill.
Hence, for a board of dimensions $2^k$ where k $>= 0$, we can fill everything but a single cell just using L shapes. \\

At the merge step, we are combining 4 similar subproblems.
However, only one subproblem has the cell taken out, while the other 3 do not.
Hence, we still have 3 subproblems where they have a single cell not filled out. 
If we were to orientate these 3 subproblems such that 
the missing cells are at the corners of the grid, 
we are actually able to combine these 3 missing cells towards the middle during merging.
Hence, these 3 missing cells can actually be fitted with a final L such that we can get a complete grid given that we remove a single cell.
Therefore, at any merging step, we are able to combine subproblems to form a valid solution.

\newpage

\section*{Problem 2}

\lstset{caption={Counting Inversions, Merge Sort}}
    \lstset{label={lst:alg1}, numbers=left}
    \begin{lstlisting}[style = Python]
        def count_inversions(arr: list) -> int:
            def combine(a1: list, a2: list) -> (int, list):
                inversions = 0
                i, j = 0, 0
                # we sort the two input arrays and return it
                sorted_arr = []

                # iterate using two pointers to compare since they are sorted 
                while i < len(a1) and j < len(a2):
                    if a1[i] <= a2[j]:
                        sorted_arr.append(a1[i])
                        i += 1
                    else:
                        # invariant that both arrays are sorted
                        # hence, all numbers after a1[i] will cause an inversion
                        inversions += (len(a1) - i)
                        sorted_arr.append(a2[j])
                        j += 1
        
                # fill in the rest of the array
                while i < len(a1):
                    sorted_arr.append(a1[i])
                    i += 1
                while j < len(a2):
                    sorted_arr.append(a2[j])
                    j += 1
                    
                return inversions, sorted_arr
            
            def merge(arr: list) -> (int, list):
                # base case, nothing more to split
                if len(arr) <= 1:
                    return 0, arr
                
                mid = len(arr) // 2

                left_inversions, lhs = merge(arr[:mid])
                right_inversions, rhs = merge(arr[mid:])
                inversions, sorted_arr = combine(lhs, rhs)
                # sum up the inversions we got from our subproblems
                res = left_inversions + right_inversions + inversions
                
                return res, sorted_arr
            
            return merge(arr)[0]
  \end{lstlisting}

    We can use divide and conquer to solve this problem, applying a similar technique to merge sort.
    Instead of dealing with an array of size n, we can instead break up the array into two halves of similar length n/2. 
    We keep breaking it up until we have small chunks of size 1 or an empty array, which is our base case.
    Once we have obtained our small arrays, we can then do the rebuilding/ combining step, similar to merge sort. \\
    
    In this step, we can sort our array and also count the inversions at the same time.
    This is because it is easier to count the number of inversions when we have 2 sorted arrays.
    For example, given a1[i] and a2[j] such that a1[i] $>$ a2[j].
    If this was true, then any element after index i in a1 will also require an inversion.
    Hence, we can exploit this property to add (len(a1) - i) inversions at once (the remaining number of elements greater than a1[i]).
    We can then return our merged array which is sorted as well as the total number of inversions from this combination step.
    It will then be propagated for use in upper level recursion calls. \\
    
    Invariants: 
    Arrays in combination step are sorted, maintained by the fact that it the combination step also returns sorted arrays. \\
    Additionally, due to the invariant above, there are (len(a1) - i) additional inversions for a single comparison a[i] $>$ a[j] if it holds true. \\
    
    Runtime: 
    For a problem size of n, we split it into 2 subproblems of size n/2 recursively. 
    This gives the relationship of subproblems to be T(n) = 2T(n/2).
    For the merging step, we iteratively go through the subarrays of combined sizes n, n/2, n/4, ..., n/k such that n/k = 1.
    Hence, at each an arbitrary level k, we do O(n/k) amount of work for the merge. 
    Thus, the final recurrence can be given by T(n) = 2T(n/2) + O(n). 
    Therefore, at each level, we do O(n) amount of work.
    Given that there are $log_2 n$ levels, the time complexity is hence bounded by O(nlogn).

\newpage

\section*{Problem 3}

\begin{enumerate}
    \item 
    \lstset{caption={Best Subset Middle (BSM)}}
    \lstset{label={lst:alg1}, numbers=left}
    \begin{lstlisting}[style = Python]
        def bsm(arr: list) -> int:
            mid = len(arr) // 2
            left_max, left_curr = float("-inf"), 0
            right_max, right_curr = float("-inf"), 0
            
            for i in range(mid, -1, -1):
                left_curr += arr[i]
                left_max = max(left_max, left_curr)
            
            for i in range(mid + 1, len(arr)):
                right_curr += arr[i]
                right_max = max(right_max, right_curr)

            # add the two prefix sums
            return left_max + right_max
    \end{lstlisting}
    
    Let us use a prefix sum algorithm to solve the problem of best subset middle (BSM). 
    BSM states that we just have to find a range of values which includes our floor(n/2) element 
    such that the sum of these values is maximum.
    We can start by pivoting around our middle element and
    searching for contiguous subarrays on the LHS and RHS of the middle element that has maximal sum.
    As our final result has to be a contiguous subarray, 
    we can use a prefix sum technique of accumulating the values. 
    Using a prefix sum, we don't consider a single element by itself, 
    but instead iterate and accumulate the values.
    What we care about instead is whether or not it contributes to our accumulated value, and makes it larger.
    If it does, it means that we can extend our contiguous subarray.
    After finding the contiguous subarrays on the LHS and RHS, 
    we can then sum them up as both subarrays give the maximal sums on their respective sides. \\
    
    Invariant: 
    The total we get from accumulating a set of values 
    guarantees us that all elements up till this index is used in the accumulation. \\

    Runtime: 
    As we only need to iterate the LHS and RHS of the median once, it would be 2 separate O(n/2) operations, hence it is O(n) overall.

    \newpage
    
    \item 
    \lstset{caption={Best Subset (BS)}}
    \lstset{label={lst:alg1}, numbers=left}
    \begin{lstlisting}[style = Python]
        def bs(arr: list) -> int:
            def bsm(arr: list) -> int:
                # from previous part ...
            
            def merge(arr: list) -> int:
                if len(arr) == 0:
                    return float("-inf")
                if len(arr) == 1:
                    return arr[0]
                
                mid = (len(arr)) // 2
                left_merge_max = merge(arr[:mid])
                right_merge_max = merge(arr[mid:])
                bsm_curr_arr = bsm(arr)
                
                return max(left_merge_max, right_merge_max, bsm_curr_arr)
                
            return merge(arr)
    \end{lstlisting}
    
    Let us call our function from the previous section BSM. Given an input array of size n, 
    we can find the best subset (BS) using a technique similar to merge sort.
    Let us recursively split our sized array n into 2 halves of n/2.
    We can repeatedly do so until we have a base case of array sized 1 or an empty array.
    Once we have gotten our small arrays, we can do the rebuilding/ combining step which utilizes our BSM. 
    Hence, here we have 3 values to compare with: either the max from the LHS, max from RHS, 
    or the new max from utilizing some subarray that crosses the middle.
    We require the BSM step as when we find the max on the LHS and RHS, 
    we don't consider if there could be a higher max from some combination of the LHS and RHS
    (as we consider them separately while doing the divide and conquer).

    \item 
    There are 3 cases in this problem and our algorithm. 
    We can either take the max from the LHS, the RHS or the BSM.
    Let us assume from the previous part that we have proven the correctness of BSM. 
    In that case, we can assume that BSM does give us an accurate max. \\
    
    Invariant:
    As our function call of merge returns the max of all 3 options, we can thus guarantee that similar recursive function calls will also return the maximum. \\
    
    Runtime: 
    For a problem size of n, we split it into 2 subproblems of size n/2 recursively. 
    This gives the relationship of subproblems to be T(n) = 2T(n/2).
    For the merging step (BSM), we iteratively go through the subarrays of combined sizes n, n/2, n/4, ..., n/k such that n/k = 1.
    Hence, at each an arbitrary level k, we do O(n/k) amount of work for the merge. 
    Thus, the final recurrence can be given by T(n) = 2T(n/2) + O(n). 
    Therefore, at each level, we do O(n) amount of work.
    Given that there are $log_2 n$ levels, the time complexity is hence bounded by O(nlogn).

    \item 
    \lstset{caption={Best Subset (BS) Linear}}
    \lstset{label={lst:alg1}, numbers=left}
    \begin{lstlisting}[style = Python]
        def bs_linear(arr: list) -> int:
            l = 0
            total, res = 0, float("-inf")
        
            # two pointer approach
            for r, val in enumerate(arr):
                # reset value when negative
                while total < 0 and l < r:
                    total -= arr[l]
                    l += 1
                total += val
                res = max(res, total)
                
            return res
    \end{lstlisting}

    For a linear time algorithm, we can use a two pointer approach.
    The two pointer approach maintains our contiguous subarray.
    We can keep adding elements and updating the total until it becomes too small.
    We define too small as a negative total, 
    where at that point we can try to remove some of the earlier numbers 
    (imagine an array with all negative elements).
    We continue this process until we reach the end of the array. \\

    Invariants:
    This applies the prefix sum technique. 
    Thus at every index i, we can guarantee that we have found the maximum possible sum for a subset using elements up to i inclusive. \\

    Runtime:
    O(n) as we are just iterating the array from left to right.

\end{enumerate}

\end{document}
