\documentclass[12pt,letterpaper]{article}
\usepackage{fullpage}
\usepackage[top=2cm, bottom=4.5cm, left=2.5cm, right=2.5cm]{geometry}
\usepackage{amsmath,amsthm,amsfonts,amssymb,amscd}
\usepackage{lastpage}
\usepackage{enumerate}
\usepackage{fancyhdr}
\usepackage{mathrsfs}
\usepackage{xcolor}
\usepackage{graphicx}
\usepackage{listings}
\usepackage{hyperref}
\usepackage{amsmath}
\usepackage{mathtools}
\usepackage{tikz}
\usepackage{array}
\usetikzlibrary{matrix}

\hypersetup{%
  colorlinks=true,
  linkcolor=blue,
  linkbordercolor={0 0 1}
}
 
\renewcommand\lstlistingname{Section}
\renewcommand\lstlistlistingname{Algorithms}
\def\lstlistingautorefname{Alg.}

\lstdefinestyle{Python}{
    language        = Python,
    frame           = lines, 
    basicstyle      = \footnotesize,
    keywordstyle    = \color{blue},
    stringstyle     = \color{green},
    commentstyle    = \color{red}\ttfamily
}

\setlength{\parindent}{0.0in}
\setlength{\parskip}{0.05in}

% Edit these as appropriate
\newcommand\course{CSE 3500}
\newcommand\hwnumber{6}                  % <-- homework number
\newcommand\NetIDa{rjf23002}           % <-- NetID of person #1
\newcommand\NetIDb{}           % <-- NetID of person #2 (Comment this line out for problem sets)

\pagestyle{fancyplain}
\headheight 35pt
\lhead{\NetIDa}
\lhead{\NetIDa\\\NetIDb}                 % <-- Comment this line out for problem sets (make sure you are person #1)
\chead{\textbf{\Large Homework \hwnumber}}
\rhead{\course \\ \today}
\lfoot{}
\cfoot{}
\rfoot{\small\thepage}
\headsep 1.5em

\begin{document}

\section*{Problem 0}
\begin{enumerate}
  \item
    No.
    A scheduling solution cannot be used to solve a node cover solution. 
    A scheduling solution is greedy in nature and chooses nodes which give it the most profit P
    in a certain time t.
    However, this selection of nodes may not ensure that the node cover property is adhered too,
    i.e it doesn't guarantee us that we select the smallest selection of nodes,
    but rather just a set of nodes based on its greedy approach. 
  \item
    Yes.
    A node cover solution can use be to use solve a scheduling problem of at least profit P.
    A node cover solution just chooses a smallest set of nodes which are able to be connected to all other nodes through its edges.
    As such, if we were to formulate the scheduling problem as a graph, 
    we can imagine at each step we can choose a job i, and following that we pick other jobs which are directly connected (excluding i).
    Hence, a node cover solution would produce a solution which is able to maximize connectivity among the graph.
    From there, we have certain jobs selected from the node cover.
    What's left is to see if there are any other remaining jobs that have not gone past the deadline, 
    and if the total sum of points is able to exceed P.
\end{enumerate}

\section*{Problem 1}
Hamilton Path $>$ Hamilton cycle: \\
Not every Hamilton path is a Hamilton cycle. 
Given that a Hamilton cycle definitely exists in a graph, 
we can generate all valid Hamilton paths.
From there, we can then choose a particular path where it is just one edge away from becoming a cycle. \\\\
Hamilton Path $<$ Hamilton cycle: \\
If we can find a valid cycle, that means that it is essentially a Hamilton path,
just without the ending loop to the vertex.

\section*{Problem 2}
We can use node cover to solve this problem.
We can construct a graph of nodes and edges where we have different types of nodes: areas and counsellors.
We form an edge between counsellors and areas.
Then, we run node cover on the different counsellors nodes to get a small
set of counsellors that cover all areas.
With our set, we can then check if we have at most k counsellors.

\section*{Problem 3}

\begin{enumerate}
  \item 
    We can design the system as a graph such that there are n process nodes.
    We then connect process nodes to each other only if they use the same resource.
    For example, if process A and B both use a resource C, then nodes A and B should have a link.
    We can then run independent set on this graph.
    If we take process A, then we cannot take process B, 
    as that would mean that process A uses resource C, 
    which disallows process B to run.
    Hence, it is consistent with the independent set problem.
    As such, we can find all independent sets possible.
    Once we have found all independent sets, we can then find if there exists one set which have a size of k.
    If so, that means that there are at least k processes that are able to run.
  \item 
    We can reuse the previous solution, but just find sets where its size is at least 2.
  \item
    We just have to modify the graph such that cpu1 and cpu2 are different entities.
    Hence, we can connect process A to both cpu1 and cpu2 if they require a cpu.
  \item 
    We can still use the solution from part a to solve this.
\end{enumerate}

\end{document}
